\chapter{Modules}

  \begin{define}
    Let $R$ be a ring. A \bt{left $\bs{R}$-module} is an abelian group
    $(M,+)$ endowed with an action of $R$, i.e. a map $R\times M \to M$,
    usually denoted $(r,m)\mapsto rm$, such that
    \begin{enumerate}
      \item $(r + r')m = rm+r'm$, $r(m+m') = rm+rm'$
      \item $1m = m$
      \item $(rs)m = r(sm)$
    \end{enumerate}
  \end{define}
  This is the same data as a ring homomorphism $R \to \End_\ZZ M$.

  Examples:
  \begin{enumerate}
    \item $\{0\}$. $r\cdot 0 = 0$
    \item $\sub{R}{R}$ where $r\cdot m = rm$.
    \item If $M$ is a left $R$-module, it becomes a right $R^{\op}$-module by
      $mr^o = rm$.
  \end{enumerate}

  \begin{prop}
    A module over a field $k$ is a $k$-vector space.
  \end{prop}

  \begin{define}
    A subgroup $N$ of a left $R$-module $M$ is a \bt{submodule} if $rn\in N$
    for all $r\in R$ and $n\in N$.
  \end{define}

  The submodules of $\sub{R}{R}$ are the left ideals of $R$.

  \begin{define}
    A non-zero left $R$-module $M$ is \bt{simple} if its only submodules are
    $\{0\}$ and $M$ itself.
  \end{define}

  \begin{define}
    If $M$ and $N$ are left $R$-modules, a map $f:M\to N$ is a
    \bt{homomorphism} of left $R$-modules if it is a group homomorphism
    and $f(rm) = rf(m)$ for all $r\in R$ and $m\in M$.
  \end{define}

  \begin{define}
    A homomorphism $f$ of $R$-modules is an \bt{isomorphism} if it is
    bijective.
  \end{define}
  The bijectivity allows us to define $f^{-1}$ and $f^{-1}$ is an $R$-module
  homomorphism.

  \begin{define}
    If $N$ is a submodule of $M$, then $M/N$ becomes a left $R$-module via
    \[ r(m + N) = rm + N, \]
    we call this a \bt{quotient module}.
  \end{define}
  Observe that if $m-m'\in N$, then $rm - rm'\in N$ for all $r\in R$. So
  $r(m+N)$ is well-defined.

  More module examples:
  \begin{enumerate}
    \item $R/I$ where $I$ is a left ideal of $R$.
    \item If $M$ and $N$ are left $R$-modules, their direct sum
      \[ M\oplus N = \{(m,n): m \in M, n\in N\} \]
      is a left $R$-module via $r(m,n) = (rm,rn)$.
    \item free $R$-modules of finite rank: these are
      \[ R^n = \underbrace{R \oplus \cdots \oplus R}_{n\text{ times}}. \]
      If $I_1,\ldots,I_n$ are left ideals of $R$, then
      $I_1\oplus \cdots \oplus I_n$ is a submodule of $R^n$.
    \item Let $R$ be the ring of differential operators with $\CC$-valued
      polynomials as coefficients. Then $\CC[x]$ is a left $R$-module via
      $D \cdot f = D(f)$. e.g. let $D=(1-x^2)\frac{d^2}{dx^2}+2x\frac{d}{dx}$
      and $f(x) = 2x-1$, then $D(f) = 4x$.
  \end{enumerate}

  \begin{lemma}
    If $f:M\to N$ is an $R$-module homomorphism, then $\ker f$ is a submodule
    of $M$, the image of $f$ is a submodule of $N$ and $M/\ker f \cong \im f$
    as $R$-modules via $m+\ker f \mapsto f(m)$.
  \end{lemma}

  \begin{define}
    If $M$ is a left $R$-module, we write $\End_R{M} = \{f:M\to M\}$, this is
    called the \bt{Endomorphism Ring} of $M$.
  \end{define}

  \begin{lemma}[Schur's Lemma]
    If $M$ is a simple $R$-module, then $\End_R{M}$ is a division ring.
  \end{lemma}

  \begin{prop}
    If $M$ is a left $R$-module, there is a ring homomorphism
    $\psi:Z(R)\to\End_R{M}$ given by $\psi(z)(m)=zm$.
  \end{prop}

  \begin{cor}
    If $R$ is a $k$-algebra and $M$ a left $R$-module, then there is a
    homomorphism $k \to\End_R{M}$.
  \end{cor}

  \begin{thm}[Dixmier's Theorem]
    If $R$ is a $\CC$-algebra of countable dimension, and $M$ is a simple
    $R$-module, then the canonical map $\psi:\CC \to \End_R{M}$ is an
    isomorphism.
  \end{thm}

  \begin{lemma}
    If $I$ is a left ideal in $R$ and $I\neq R$, then there is a maximal
    left ideal that contains $I$.
  \end{lemma}
  \begin{proof}
    Let $\Ac$ be the set of left ideals that contain $I$ and are not $R$, then
    there is a sequence $I_1 \subset I_2 \subset \cdots$ of $I_j\in \Ac$. But
    $\bigcup I_j$ is a left ideal, which is maximal, containing $I$.
  \end{proof}

  \begin{prop}\label{mod:max_ind}
    If $R$ has a unique maximal left ideal, $\sub{R}{R}$ is indecomposable.
  \end{prop}
  \begin{proof}
    Let $J$ be the unique maximal left ideal, then if $R = L+N$ for some left
    ideals $L$ and $N$, and $L\neq R$, then $L\subseteq J$. Hence
    $N\not\subseteq J$, and so $N=R$, which implies that  if $L\neq 0$,
    $L\cap N \neq 0$.
  \end{proof}

  \begin{prop}
    Let $R=k[x_1,\ldots,x_n]$ and let $m=(x_1,\ldots,x_n)$. If $I\supseteq m^d$
    for some $d$, then $R/I$ is an indecomposable $R$-module (because it has a
    unique maximal ideal, namely $m/I$).
  \end{prop}
  \begin{proof}
    Every maximal ideal of $R/I$ is of the form $n/I$ for some maximal ideal
    $n$ of $R$ containing $I$. Now $R/n$ is a simple $R$-module and
    $I\cdot(R/n) = (I+n)/n = n/n = 0$, but $m^d \subseteq I$, so $m^d(R/n)=0$.
    If $S$ is a simple left $R$-module and $J$ is any left ideal of $R$, then
    $JS$ is a submodule of $S$ so either $JS = 0$ or $JS = S$. However, if
    $JS=S$, then $J^dS = S$ for all $d$, so $m(R/n) = 0$. But
    $m(R/n) = (m+n)/n$ so $m\subseteq n$, which implies $m = n$ since $m$ and
    $n$ are both maximal ideals. By Proposition \ref{mod:max_ind}, $R/I$ is
    an indecomposable $R/I$-module and hence an indecomposable $R$-module.
  \end{proof}

