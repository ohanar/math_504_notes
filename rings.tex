\chapter{Rings}

\section{The Basics}

  \begin{define}
    A communtative ring $k$ is a \bt{field} if $1 \neq 0$ and if
    $x\in k-\{0\}$, then $\exists x^{-1} \in k$ such that $x^{-1}x=1$.
  \end{define}

  Examples: $\RR, \QQ, \CC, \FF_p = \ZZ/p = \ZZ/(p)$

  \begin{define}
    A non-empty subset $I$ of a ring $R$ a \bt{left ideal} if
    \begin{enumerate}
      \item $a+b \in I$ for all $a,b \in I$
      \item $xa \in I$ for all $x \in R$ and $a \in I$.
    \end{enumerate}
  \end{define}

  i.e. $I$ is a subgroup of $(R, +)$ and $xI \subset I$ for all $x \in R$.

  Similar for right ideal.

  \begin{define}
    If $I$ is a left and a right ideal it is called a \bt{two-sided ideal}.
    (Often just called an ideal).
  \end{define}

  \begin{define}
    If $R$ is a ring and then we call $aR$ the \bt{principal right ideal}, the
    ideal generated by $a$:
    \[ aR = \{ ax : x \in R\} \]
  \end{define}

  If $R$ is communtative, write $(a)$ for $aR = Ra$.

  \begin{define}
    We define quotients $R/I$ in the usual way. Its elements are cosets:
    \[ x + I = \{ x + a : a \in I \} \]
  \end{define}

  Example: $\ZZ/(p)$ elts are $0 + (p), 1 + (p), \ldots, p-1 + (p)$.

  \begin{lemma}
    If $I$ is a two-sided ideal, then $R/I$ is a ring with
    $(a+I) + (b+I) = a+b+I$ and $(a+I)(b+I) = ab + I$.
  \end{lemma}

  \begin{define}
    Let $R$ be communtative. An ideal $m$ is \bt{maximal} if $m \neq R$
    and if $I$ is an ideal such that $m \subset I \subset R$, then $I = m$ or
    $I = R$.
  \end{define}

  \begin{prop}\label{rings:quo_bij}
    Let $I$ be an ideal in a ring $R$. There is a bijection between ideals in
    $R$ that contain $I$ and ideals in $R/I$.
  \end{prop}

  Bijection: $ J \longleftrightarrow J/I \subset R/I $

  Example: Let $I$ be an ideal in $R$, the map $f: R \to R/I$,
  $f(a) = a + I$, is a ring homomorphism.

  \begin{define}
    The \bt{kernel} of a homomorphism $f:R\to S$ is the set of elements in
    $R$ that are mapped to $0_S$. i.e.
    \[ \ker f = \{ a \in R : f(a) = 0_S \} \]
  \end{define}

  \begin{define}
    A homomorphism $g: R_1 \to R_2$ is an \bt{isomorphism} if it is a
    bijective.
  \end{define}

  Note: $g^{-1}$ is also an isomorphism.

  \begin{prop}
    Let $f: R \to S$ be a ring homomorphism, then there is a ring isomorphism
    $\bar{f}: R/\ker{f} \to \im{f}$, given by
    \[ \bar{f}(a+\ker{f}) = f(a). \]
  \end{prop}

  \begin{proof}
    Check that $\ker{f}$ is a two-sided ideal. Certainly it is a subgroup of
    $(R, +)$ (it is a group homomorphism). If $x \in R$, $a \in \ker{f}$, then
    \[f(ax) = f(a)f(x) = 0\cdot f(x) = 0.\]
    Same for other side.
  \end{proof}

  \begin{lemma}
    A non-trivial communtative ring $R$ is a field if and only if the only
    ideals in $R$ are $\{0\}$ and $R$.
  \end{lemma}

  \begin{proof}
    Suppose $I$ is an ideal in a field $R$, if $I \neq \{0\}$, then there is
    an element $a\in I-\{0\}$. Now $a^{-1} \in R$, so $a^{-1}a = 1 \in I$. So
    $x\cdot 1 \in I$ for all $x \in R$, so $I = R$.

    Conversely, suppose that the only ideals are $\{0\}$ and $R$. Let
    $0 \neq a \in R$, then $aR$ is an ideal that is not equal to $\{0\}$, so
    $aR = R$. Thus there exists $b \in R$ such that $ab = 1 \in aR = R$.
  \end{proof}

  \begin{prop}
    If $R$ is a communtative ring and $m$ a maximal ideal, then $R/m$ is a
    field.
  \end{prop}

  \begin{proof}
    We know that only $\{m,R\}$ contain $m$, then from prop
    \ref{rings:quo_bij} we know $R/m$ has two ideals $\{R/m, m/m = \{0\}\}$,
    so $R/m$ is a field.
  \end{proof}

  \begin{lemma}
    Every ideal in $\ZZ$ is principal.
  \end{lemma}

  \begin{proof}
    Let $I$ be a non-zero ideal and let $d$ be the smallest positive integer
    in $I$, and note that $d\ZZ \subset I$. If $a \in I$, then $a = dq + r$
    for some $q \in \ZZ$ and $r \in \{0, 1, \ldots, d-1\}$. Whence
    $r = a-dq \in I$, so $r = 0$ by choice of $d$, hence $a = dq \in \ZZ$.
  \end{proof}

  \begin{prop}
    If $p\in \ZZ$ is prime then $p\ZZ$ is maximal, every maximal ideal of $\ZZ$
    is of this form, and $\ZZ/p\ZZ$ is a field.
  \end{prop}

  \begin{lemma}
    If $R$ is a ring with identity, it has a maximal ideal, a maximal left, and
    a maximal right ideal.
  \end{lemma}
  \begin{proof}
    Let $A = \{\{0\}\} \cup \{ $I$ : 1 \in I\}$. If
    $I_1 \subset I_2 \subset \cdots$ is an ascending chain of ideals in $A$,
    then
    \[ J = \bigcup_{j=1}^\infty I_j \in A. \]
    If $a,b \in J$, then there exists $n$ such that $a,b \in I_n$, whence
    $a + b \in I_n$, and $ax,xa \in I_n$ for $x \in R$. Hence $J$ is an ideal
    and $J$ in $A$. So $A$ has upper bounds, so by Zorn it has maximal members.
    Such a maximal member of $A$ is a maximal ideal of $R$.
  \end{proof}

  \begin{cor}
    When $R$ is commutative then there exists a surjective homomorphism
    $R \to k$, for some field $k$.
  \end{cor}

  \begin{cor}
    Every ideal $I \neq R$ is contained in a maximal ideal.
  \end{cor}

  \bt{Warning:} It is not the case that every left ideal is contained in a
  maximal two-sided ideal. e.g. Let $R = M_2(\CC)$, and consider
  \[ I = \left\{ \begin{bmatrix} a & 0 \\ b & 0 \end{bmatrix} : a,b\in \CC \right\}, \]
  but the only two-sided ideals in $R$ are $R$ and $\{0\}$.

  \begin{define}
    If the only two-sided ideals are $\{0\}$ and $R$, then we call $R$ a
    \bt{simple ring}.
  \end{define}

\section{Algebras}

  \begin{define}
    Let $k$ be a field, then a $\bs{k}$\bt{-algebra} is a $k$ vector space $A$
    that is a ring in which multiplication $\mu: A \times A \to A$ is
    $k$-bilinear. i.e.
    \[\mu(\lambda a + \gamma b, c) = \lambda\mu(a,c) + \gamma\mu(b,c) \]
    etc.
  \end{define}

  Examples: $k, M_n(k), \begin{bmatrix} k & k \\ 0 & k \end{bmatrix},
    \begin{bmatrix} k & 0 \\ k & k \end{bmatrix}, k[x_1,\cdots,x_n]$

  \begin{define}
    If $G$ is a group, let $kG$ be the vector space with basis $\{g \in G\}$.
    I.e. an element of $kG$ is a finite formal sum
    \[ \sum_{g \in G} \lambda_g \]
    where $\lambda_g \in k$, and multiplication is the bilinear extension of
    \[ (\lambda_g g) (\lambda_h h) = (\lambda_g \lambda_h) (g h). \]
    We call this type of $k$-algebra a \text{Group algebra}.
  \end{define}

  \begin{define}
    We define a \bt{Free algebra} $k\langle x_1,\ldots,x_n\rangle$
    to be a $k$-algebra whose basis is all words in $x_1,\ldots,x_n$,
    and multiplication is the bilinear extension of word concatenation.
    i.g. $(xyz)(zx) = xyzzx$.
  \end{define}

  Example: Consider $k\langle x\rangle$, the basis is
  \[ \emptyset, x, xx, xxx, \ldots\]
  But this is just the $k\langle x \rangle \cong k[x]$.

  Notice that $k\langle x, y\rangle$ is not commutative because $xy \neq yx$.
  In fact $k\langle x, y\rangle$ contains a subalgebra isomorphic to
  $k\langle x_1, \ldots, x_n\rangle$ for every $n$. Check that the subalgebra
  $k\langle x, xy, xy^2, \ldots xy^{n-1} \rangle$ is isomorphic to
  $k\langle x_1, \ldots, x_n \rangle$.
