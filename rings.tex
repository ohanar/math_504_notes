\documentclass{article}

	\usepackage[margin=1in]{geometry}
	\usepackage{amsmath}
	\usepackage{amssymb}
	\usepackage{amsthm}
	
	\newcommand{\NN}{\mathbb{N}}
	\newcommand{\ZZ}{\mathbb{Z}}
	\newcommand{\QQ}{\mathbb{Q}}
	\newcommand{\HH}{\mathbb{H}}
	\newcommand{\RR}{\mathbb{R}}
	\newcommand{\CC}{\mathbb{C}}
	\newcommand{\FF}{\mathbb{F}}

	\renewcommand{\theenumi}{(\alph{enumi})}
	\renewcommand{\labelenumi}{\theenumi}

    \newtheorem{define}{Definition}
    \newtheorem{prop}{Proposition}
    \newtheorem{lemma}{Lemma}

    \DeclareMathOperator{\im}{im}
	
\begin{document}

\section{Rings}

\subsection{Definition}

  \begin{define}
    A communtative ring $k$ is a \textbf{field} if $1 \neq 0$ and if
    $x\in k-\{0\}$, then $\exists x^{-1} \in k$ such that $x^{-1}x=1$.
  \end{define}

  Examples: $\RR, \QQ, \CC, \FF_p = \ZZ/p = \ZZ/(p)$

  \begin{define}
    A non-empty subset $I$ of a ring $R$ a \textbf{left ideal} if
    \begin{enumerate}
      \item $a+b \in I$ for all $a,b \in I$
      \item $xa \in I$ for all $x \in R$ and $a \in I$.
    \end{enumerate}
  \end{define}

  i.e. $I$ is a subgroup of $(R, +)$ and $xI \subset I$ for all $x \in R$.

  Similar for right ideal.

  \begin{define}
    If $I$ is a left and a right ideal it is called a \textbf{two-sided ideal}.
    (Often just called an ideal).
  \end{define}

  \begin{define}
    If $R$ is a ring and then we call $aR$ the \textbf{principal right ideal}, the
    ideal generated by $a$:
    \[ aR = \{ ax : x \in R\} \]
  \end{define}

  If $R$ is communtative, write $(a)$ for $aR = Ra$.

  \begin{define}
    We define quotients $R/I$ in the usual way. Its elements are cosets:
    \[ x + I = \{ x + a : a \in I \} \]
  \end{define}

  Example: $\ZZ/(p)$ elts are $0 + (p), 1 + (p), \ldots, p-1 + (p)$.

  \begin{lemma}
    If $I$ is a two-sided ideal, then $R/I$ is a ring with
    $(a+I) + (b+I) = a+b+I$ and $(a+I)(b+I) = ab + I$.
  \end{lemma}

  \begin{define}
    Let $R$ be communtative. An ideal $m$ is \textbf{maximal} if $m \neq R$
    and if $I$ is an ideal such that $m \subset I \subset R$, then $I = m$ or
    $I = R$.
  \end{define}

  \begin{prop}\label{rings:quo_bij}
    Let $I$ be an ideal in a ring $R$. There is a bijection between ideals in
    $R$ that contain $I$ and ideals in $R/I$.
  \end{prop}

  Bijection: $ J \longleftrightarrow J/I \subset R/I $

  Example: Let $I$ be an ideal in $R$, the map $f: R \to R/I$,
  $f(a) = a + I$, is a ring homomorphism.

  \begin{define}
    The \textbf{kernel} of a homomorphism $f:R\to S$ is the set of elements in
    $R$ that are mapped to $0_S$. i.e.
    \[ \ker f = \{ a \in R : f(a) = 0_S \} \]
  \end{define}

  \begin{define}
    A homomorphism $g: R_1 \to R_2$ is an \textbf{isomorphism} if it is a
    bijective.
  \end{define}

  Note: $g^{-1}$ is also an isomorphism.

  \begin{prop}
    Let $f: R \to S$ be a ring homomorphism, then there is a ring isomorphism
    $\bar{f}: R/\ker{f} \to \im(f)$, given by
    \[ \bar{f}(a+\ker{f}) = f(a). \]
  \end{prop}

  \begin{proof}
    Check that $\ker{f}$ is a two-sided ideal. Certainly it is a subgroup of
    $(R, +)$ (it is a group homomorphism). If $x \in R$, $a \in \ker{f}$, then
    \[f(ax) = f(a)f(x) = 0\cdot f(x) = 0.\]
    Same for other side.
  \end{proof}

  \begin{lemma}
    A non-trivial communtative ring $R$ is a field if and only if the only
    ideals in $R$ are $\{0\}$ and $R$.
  \end{lemma}

  \begin{proof}
    Suppose $I$ is an ideal in a field $R$, if $I \neq \{0\}$, then there is
    an element $a\in I-\{0\}$. Now $a^{-1} \in R$, so $a^{-1}a = 1 \in I$. So
    $x\cdot 1 \in I$ for all $x \in R$, so $I = R$.

    Conversely, suppose that the only ideals are $\{0\}$ and $R$. Let
    $0 \neq a \in R$, then $aR$ is an ideal that is not equal to $\{0\}$, so
    $aR = R$. Thus there exists $b \in R$ such that $ab = 1 \in aR = R$.
  \end{proof}

  \begin{prop}
    If $R$ is a communtative ring and $m$ a maximal ideal, then $R/m$ is a
    field.
  \end{prop}

  \begin{proof}
    We know that only $\{m,R\}$ contain $m$, then from prop
    \ref{rings:quo_bij} we know $R/m$ has two ideals $\{R/m, m/m = \{0\}$,
    so $R/m$ is a field.
  \end{proof}

  \begin{lemma}
    Every ideal in $\ZZ$ is principal.
  \end{lemma}

  \begin{proof}
    Let $I$ be a non-zero ideal and let $d$ be the smallest positive integer
    in $I$, and note that $d\ZZ \subset I$. If $a \in I$, then $a = dq + r$
    for some $q \in \ZZ$ and $r \in \{0, 1, \ldots, d-1\}$. Whence
    $r = a-dq \in I$, so $r = 0$ by choice of $d$, hence $a = dq \in \ZZ$.
  \end{proof}

  \begin{prop}
    If $p\in \ZZ$ is prime then $p\ZZ$ is maximal, every maximal ideal of $\ZZ$
    is of this form, and $\ZZ/p\ZZ$ is a field.
  \end{prop}

\end{document}
