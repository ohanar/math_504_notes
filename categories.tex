\chapter{Categories}

\section{The Basics}

  \begin{define}
    A \bt{covariant functor} $F:\Cc\to \Dc$ sending objects in $\Cc$ to objects
    in $\Dc$ and sending $\Cc(X,Y) = \{ \text{all morphisms } X\to Y\}$ to
    $\Dc(FX, FY)$ such that
    \begin{enumerate}
      \item $F(f\circ g) = F(f)\circ F(g)$
      \item $F(\Id_X) = \Id_{FX}$.
    \end{enumerate}
  \end{define}

  Example: Let $M$ be an $R$-$S$-bimodule, then
  $\Hom_R(M,\_): \Mod_l R\to \Mod_l S$ is a covariant functor if
  $f\in \Hom_R(N_1, N_2)$ then
  \[ \Hom_R(M,f): \Hom_R(M,N_1) \to \Hom_R(M, N_2) \]
  where $\alpha \mapsto f\circ \alpha$. Check $\Hom_R(M,f)$ is a homomorphism
  of left $S$-modules. Let $s \in S$, then
  $\Hom_R(M,f)(s\alpha) = f\circ s\alpha$, but
  \[ (f\circ s\alpha)(m) = f(\alpha(ms)) = (f\circ \alpha)(ms)
  = (s(f\circ\alpha))(m)\]
  so
  \[ f\circ s\alpha = s(f\circ \alpha) = s\Hom(M,f)(\alpha).\]
  Therefore $\Hom_R(M,f)$ is a homomorphism of left $S$-modules.

  \begin{define}
    A \bt{contravariant functor} $H:\Cc \to \Dc$ sends objects in $\Cc$ to
    objects in $\Dc$ and sends $\Cc(X,Y)$ to $\Dc(HY, HX)$ such that
    \begin{enumerate}
      \item $H(f\circ g) = H(g)\circ H(f)$
      \item $H(\Id_X) = \Id_{HX}$.
    \end{enumerate}
  \end{define}

  Example: If $f:\sub{R}{M}\to \sub{R}{N}$, define
  $f^*:\Hom_R(N,R) \to \Hom_R(M,R)$ to be $f^*(g) = g\circ f$
  We now have the functor $*:\Mod_l{R}\to \Mod_r{R}$
  \[ (f_1\circ f_2)^*(\alpha) = \alpha \circ f_1 \circ f_2
    = f_2^*(\alpha\circ f_1) = f_2^*(f_1^*(\alpha))
  = (f_2^*\circ f_1^*)(\alpha), \]
  so $(f_1\circ f_2)^* = f_2^*\circ f_1^*$, which means this is a contravariant
  functor.

  \begin{lemma}
    Let $0\to L\to M\to N\to 0$ be an exact sequence of left $R$-modules. Let
    $X$ be any left $R$-module. Then the sequences
    \begin{alignat*}{2}
      0&\to \Hom_R(X,L) &\to \Hom_R(X,M) &\to \Hom_R(X,N), \\
      0&\to \Hom_R(N,X) &\to \Hom_R(M,X) &\to \Hom_R(L,X)
    \end{alignat*}
    are exact.
  \end{lemma}
  \begin{proof}
    Only first case, second case is similar. Recall $f^*(\alpha)=f\circ\alpha$,
    so we must show $f^*$ is injective and $\im f^* = \ker g^*$. If
    $f^*(\alpha)=0$, then $(f\circ \alpha)(x) = 0$ for all $x \in X$, but
    $f$ is injecitive so $\alpha(x) = 0$ for all $x\in X$, so $\alpha = 0$.

    If $\alpha \in \Hom_R(X, L)$, then
    \[g^*(f^*(\alpha) = g^*(f\circ \alpha) = g\circ f\circ \alpha = 0.\]
    But $gf = 0$, so $g^*f^* = 0$ and hence $\im f^* \subseteq \ker g^*$.

    Let $\beta \in \Hom_R(X,M)$ and suppose $g^*(\beta) = 0$, so
    $(g\circ\beta)(x) = 0$ for all $x\in X$. But this means that
    $\beta(x) \in \ker g$, so there is a unique $l\in L$ such that
    $f(l) = \beta(x)$, so we can define a map $\gamma:X\to L$ by
    making $\gamma(x)$ be the unique element in $L$ such that
    $f(\gamma(x)) = \beta(x)$.

    If $r\in R$, then
    \[ f(\gamma(rx)) = \beta(rx) = r\beta(x) = r f(\gamma(x)) = f(r\gamma(x)), \]
    but $f$ is injective so $\gamma(rx) = r\gamma(x)$, so
    $\gamma \in \Hom_R(X,L)$. Therefore
    $\beta = f\circ \gamma = f^*(\gamma)$, so $\ker g^* \subseteq \im f^*$.
  \end{proof}

  Example: Consider $0\to \ZZ/2 \too{f} \ZZ/4 \too{g} \ZZ/2 \to 0$ with
  $f(x) = 2x$ and $g(x) = x + \ZZ/2$. Suppose that there es a homomorphism
  $\delta:\ZZ/2\to \ZZ/4$ such that $\Id = g\circ \delta = g^*(\delta)$,
  then $\delta \neq 0$. But there is only one non-zero group homomorphism
  from $\ZZ/2\too{g} \ZZ/4$, namely $f$. So $g\circ\delta=g\circ f=0\neq \Id$,
  hence $\Hom_\ZZ(\ZZ/2,\ZZ/4)\too{g^*} \Hom_{\ZZ}(\ZZ/2,\ZZ/2)$ is not
  surjective.

  \begin{define}
    If $F:\Mod R\to \Mod S$ sends short exact sequences to short exact
    sequences, call $F$ an \bt{exact functor}.
  \end{define}

  Example: $\Hom_R(X,\_)$ is not always an exact functor. $H$ is a left left
  exact.

  \begin{prop}\label{cat:proj}
    Let $P$ be a left $R$-module, then the following are equivalent:
    \begin{enumerate}
      \item $\Hom_R(P,\_)$ is an exact functor.
      \item There exists an $R$-module $Q$ such that $P\oplus Q$ is a free
        $R$-module.
      \item Given a surjective $R$-module homomorphism $g:M\to N$ and a
        homomorphism $\delta:P\to N$ and $\beta:P\to M$ such that
        $\delta = g\circ \beta$.
    \end{enumerate}
  \end{prop}

  \begin{define}
    A left $R$-module $P$ is \bt{projective} if it satisfies any of the
    conditions in Proposition \ref{cat:proj}.
  \end{define}
